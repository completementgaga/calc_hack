\documentclass[12pt]{article}
\usepackage{moodlept}
\usepackage[utf8]{inputenc}
\usepackage[T1]{fontenc}

\begin{document}

\begin{quiz}{An example of a quiz with calculated questions}
\begin{numerical}[points=2]{Basic addition, calculated}
  What is ${a}+{b}$?
  \item {a}+{b}
  \end{numerical}
  params\\
  a: range(19)\\
  b: -7,1,2,3,4,5,6,7,8,9,10,11,14,15,17,18,19,20,21,22,23,24,27,28,29,-8\\
  digits(for the result)\\
  1

  \begin{multi}{estudo de uma sequência}
  Seja, para todo inteiro $n>0$, \[v_n=\frac{e^n}{n}.\]
  A sequência $(v_n)$ possui o limite
  \item $1$.
  \item $0$.
  \item $e$.
  \item* $+\infty$.
  \end{multi}


\begin{multi}[multiple]{Sequências, Convergência}
A sequência $\displaystyle \frac{n^2}{{b}} \text{sen}\left(\frac{{a}}{n^{2}}\right)$
\item* Converge a {a}/{b}
\item diverge, pois fica oscilando com valores positivos e negativos.
\item tem limite igual a infinito
\item Converge a 0.
\end{multi}

params\\
a: range(1,9)\\
b: 1,2,3\\
digits(for the calculated results)\\
0
\end{quiz}
\end{document}
%%%%%%%%%%%%%%%%%%%%%%%%%%%%%
